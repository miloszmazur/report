\section{Thesis Goal}
The main goal of the thesis is to answer the question:
``How selfish and utilitarian behaviors between artificial agents influence the performance in multi-agent task-oriented environments?'' % efficiency?
\section{Thesis Research Goals}
\begin{itemize}
    \item Provide a platform that can be used to automatically generate and evaluate artificial agents.
    \item Measure and find a way to deal with the impact of the initial generation.
    \item Measure the effect cooperative behaviors had on task execution.
\end{itemize}
\section{Challenge}
The true challenge in presented work is to provide a valid way to compare two drastically different approaches to the task in order to prove whether changing said approach is tied to a change in efficiency of the task's execution. Before the grounds for the comparison can be defined, a common goal must be stated for agents in both approaches, thus formulating aforementioned task.

The technical challenge lies in handling the incorporation of Genetic Programming to the problem. Evolving Behavior Trees will be subjected to various transmutations, some of which could potentially destroy some of them. The task here is to minimize the occurrence of this happening.
\section{Thesis Structure Overview}
In order to provide the answer to a question stated in the thesis goal, following chapter will first provide essential background information on technologies used in the thesis. Chapter \ref{chapter_relatedwork} will complement that information with recent developements and related work in appropriate areas. Chapter \ref{chapter_method} will acquaint the reader with the platform the research was done on, as well as the map used in specimen evaluation, detailed description of research conditions, restraints put on the environment and actually combining the two technologies described in previous chapters. The last chapter presents the details of experiments done as part of the thesis, as well as their results.

The attempt to answer the main research question of this thesis is then presented in conclusion, along with ideas for possible future research.
