The aim of this thesis was answering the research question:  ``How cooperative behaviors between artificial agents influence the performance in multi-agent task-oriented environments?''. On the grounds of success criteria defined in the previous sections: the increase of speed in reaching the solution, the size of a successful candidate and speed of task execution itself all testify to cooperative behaviors having a considerable influence in tested task. Specimen in the cooperative scenario achieved quicker completion time compared to those evolved in the contrasting, competitive one. Furthermore, later iterations in the cooperative case would feature specimen of much smaller size executing the task with equal effectiveness.

These results prove that cooperative behaviors are, in fact, \textit{feasible} to evolve.
With that said, the work here merely touches the surface of genetically evolving Behavior Trees and the behavior types they produce. Future work will focus at deeper analysis of the population, with aim to possibly make genetic operators work with Behavior Tree representation seamlessly. Developing more sophisticated methods of crossover and mutation, aware of constraints of Behavior Tree encoding, would doubtlessly increase the effectiveness of evolving trees. Furthermore, the concept of adaptation of the agent to a given environment should be explored. The task considered in this thesis - arguably simple - has ben proven to be solvable with the simpliest of elements. A natural next step would be increasing both the dificultness of the problem as well as the resources available to the algorithm.
