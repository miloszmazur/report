%        File: report.tex
%     Created: Tue Jun 09 02:00 PM 2015 C
% Last Change: Tue Jun 09 02:00 PM 2015 C
%
\documentclass[12pt]{scrreprt}
\usepackage[utf8]{inputenc}
\usepackage [backend=bibtex]{biblatex}
\addbibresource{bibliography.bib}
\usepackage{graphicx}
\graphicspath{ {images/} }
\usepackage[autostyle]{csquotes}  
\usepackage{caption}
\usepackage{subcaption}
\begin{document}
% Title Page
\author{Miłosz Mazur}
\title{Cooperation Evolved}
\subtitle{Behavior Trees evolution by means of Genetic Programming}
\maketitle

\newpage
\chapter*{Abstract}
Behavior Trees are a method for AI programming that consists of a tree of hierarchical nodes controlling the flow of agent's decision making. They have proven, while being a pretty straightforward means to implement an AI, to be incredibly powerful way of obtaining autonomous agents, both due to a fact that the development can be iterable (one can start with implementing simple behavior and gradually improve the tree by adding and modifying nodes and branches) and allowing for, so to say, ``fallback tactics'', should the currently executed action fail.  Born in the game industry, they have since gained fair amount of popularity in other domains, including robotics.
Evolutionary algorithms, developed in John Holland’s Adaptation in Natural and Artificial Systems (1975), have been adapted for use in a vast variety of different problems, including optimisation issues and decision handling, often through introducing serious changes to both the algorithm structure and data structures used. Arguably, one of the most valuable modifications was Genetic Programming, popularised through works of John Koza (most notably Genetic Programming: On the Programming of Computers by Means of Natural Selection, 1992).

This project is an attempt to combine the two subjects, resulting in a way to automatically produce well-performing behavior trees, ready to use when autonomy and certain performance level consistency is required.

The environment used to conduct the simulations will be based on FRAIL, an in-house project of Wroclaw's Univeristy of Technology, adapted for the purposes of genetic calculations.

\chapter*{Acknowledgements}
\tableofcontents
\chapter{Introduction}
\section{Thesis Goal}
The main goal of the thesis is to answer the question:
``How selfish and utilitarian behaviors between artificial agents influence the performance in multi-agent task-oriented environments?'' % efficiency?
\section{Thesis Research Goals}
\begin{itemize}
    \item Provide a platform that can be used to automatically generate and evaluate artificial agents.
    \item Measure and find a way to deal with the impact of the initial generation.
    \item Measure the effect cooperative behaviors had on task execution.
\end{itemize}
\section{Challenge}
The true challenge in presented work is to provide a valid way to compare two drastically different approaches to the task in order to prove whether changing said approach is tied to a change in efficiency of the task's execution. Before the grounds for the comparison can be defined, a common goal must be stated for agents in both approaches, thus formulating aforementioned task.

The technical challenge lies in handling the incorporation of Genetic Programming to the problem. Evolving Behavior Trees will be subjected to various transmutations, some of which could potentially destroy some of them. The task here is to minimize the occurrence of this happening.
\section{Thesis Structure Overview}
In order to provide the answer to a question stated in the thesis goal, following chapter will first provide essential background information on technologies used in the thesis. Chapter \ref{chapter_relatedwork} will complement that information with recent developements and related work in appropriate areas. Chapter \ref{chapter_method} will acquaint the reader with the platform the research was done on, as well as the map used in specimen evaluation, detailed description of research conditions, restraints put on the environment and actually combining the two technologies described in previous chapters. The last chapter presents the details of experiments done as part of the thesis, as well as their results.

The attempt to answer the main research question of this thesis is then presented in conclusion, along with ideas for possible future research.


\chapter{Literature Review}
The idea of evolving Behavior Trees using Genetic Programming method is not an entirely new concept. Similarity in representation of the specimen (both Behavior Trees and originally proposed by J. R. Koza in \cite{kozagp} lisp programs are represented as a tree structure) allowed for nigh painless adaptation of the GP algorithm, while the popularity of Behavior Trees as the AI representation of choice grew, exceeding game programming areas and venturing into previously unexplored use cases.

Earlier implementations of this idea include evolving Behavior Tree agents for a turn strategy game DEFCON in \cite{defconbt}. In this paper, authors applied Genetic Programming method to evolve Behavior Trees acting as an artificial agent playing the game against existing AI developed by the creators of the game. The goal was to produce a tree able to beat the reference agent more then 50\% of time. Their experiments resulted with an evolved AI-bot able to beat the reference agent 55\% of the time in span of 200 games, thus proving the feasibility of the method.

Another research applied the method to evolve a controller for DelFly Explorer UAV, where the task was to find and fly through a window placed in a simulated (and later on in the research, quite real) room \cite{ksheperthesis}. In his thesis, Scheper presented and used a framework to obtain an agent with 88\% success rate in a simulated environment and 54\% in a real one. Furthermore, he was able to reencounter and describe some of the \cite{defconbt} findings: namely, low number of generations to achieve a near-optimal solution, destructive power of mutation operation etc. This was also the first time that evolutionary Behavior Trees were introduced into Robotics area of research.

Cooperation among artificial agents in multi-agent system environments is also a well-known subject, with multiple papers detailing state-of-the-art techniques \cite{cooperationstateoftheart}, \cite{cooperativeandcompetitivelearning}.
%detail kinds of games? kinds of tasks?

And lastly, there seems to be a constant, unsatiated need for more and more capable testing environments. Undoubtedly, modifying and fine-tuning features and code of the testing environment are the most sought-after features, this being the reason why so many researhers turn to in-house developed solutions. There were, however, a number of attempts of introducing a general-purpose testing environment (\cite{mason}, \cite{aisandbox}, \cite{frailweb}) or, at the very least, \textit{generalized} testing environment. Still, openess of the code and ease of modifications seem to prevail being the deciding factors. % mention familiarity with the environment?


\chapter{Environment}
\section{Introduction}
\textbf{FRAIL} (Framework for AI Laboratories) is a platform developed in-house at Wroclaw's Univeristy of Technology (PWR) for purposes of testing artificial adversaries in shooter games. \cite{frailweb}

As an in-house project, users were able to freely access and change the source code, modifying existing contollers or creating new ones. Since its launch, FRAIL has seen use as a didactic aid during Artificial Intelligence classes and bootstrap for various AI-related projects, including evaluation by a video game development and publishing company, Techland for internal use. % citation needed?

In order to fully utilize the platform's capabilities, substantial changes have been made to FRAIL's engine in one of the previous projects, allowing for the entire process of Genetic Programming to be executed on the platform itself. A Behavior Tree implementation and representation related modules were taken from current version of FRAIL and the project was reverted to one of the earlier versions to reduce bloating. Genetic Programming module was implemented as a ``building block'' that could be placed on any map operating in ``Capture the Flag'' mode. Exploiting FRAIL's scripting behaviors and usage of RTTI (Run-Time Type Identification), a number of convenience changes were introduced to control the environment and algorithm parameters.
% I'm not sure what exactly belongs here. For instance: I should note that introduction of GP to the platform was done beforehand, since I made only the GP part, it was my partner who integrated it within frail. There were no publications, however, since it was only a class project.
% furthermore, I feel the actual changes should be included in the method chapter, although ... it goes both ways, honestly.
\section{Map overview} % simulation environment overiew?
The task planned for the agents can be described as a resource retrieval. To simulate this scenario, a map resembling a warehouse has been planned and prepared for use. Figure \ref{fig:x simmaprender} presents a 3D render of a finished map, while a top-down view, featured on Figure \ref{fig:x simmaptopdown} will be referred to further describe the design.

\begin{figure}[h]
    \centering
    \includegraphics[scale=0.2]{frailmaprender}
    \caption{A rendering of scenery}
    \label{fig:x simmaprender}
\end{figure}

\begin{figure}[h]
    \centering
    \includegraphics[scale=0.2]{frailmaptopdown}
    \caption{A top-down view of the map}
    \label{fig:x simmaptopdown}
\end{figure}

The map was designed with two to four agents in mind, with the possibility of scaling up when needed. Its shape is rectangular, measuring 80 on 56 simulation units. Over the middle secion, 14 cuboid structures were placed symmetrically to act as storage containers / shelve-holding space, with access points along every edge. This placement allowed to make use of narrow alleys between containers in order to create potential conflict points, and positioning resources right next to them would hopefully allow for more faithful representation of a real-world resource gathering problem.


\chapter{Method}
\section{Motivation} % goal?
Albeit the goal stated in the introduction chapter is focused at the method of obtaining quality agents, the motivation for such endavor must be kept in mind. It would be beneficiary for a number of use cases to obtain a system whch, when provided with a sufficient description of an environment and capabilities of agents, could be used to produce ready to be implemented Behavior Trees capable of performing given assignments. Arguably, additional overhead due to implementing the environment \& agents specifications and task description must be taken into account when considering using such system, potentially making the method unfesible in scenarios when the time is of essence. Even then however, one has to weight initial investment into such project against manually designing and adjusting the AI. % is this the right place to mentioned that in some previous research capable trees were found after arguably small number of iterations?

Additionally, there is also one critical advantage to using evolutionary learning methods: the possibility of early termination through end condition. With the right fitness function the option to take an undeveloped tree as a template and improve it becomes perfectly possible. This can be useful in more sophisticated problems, when manual tuning will complement evolutionary method's solution. % to be perfectly honest, i'm not sure if this is the appropriate place to be making such claims - I mean discussing practical advantages.
\section{Formulating a Task}
In order to properly assess feasibility of the system in making, a common task had to be constructed to compare different variations of it against each other. As mentioned before, scenario type chosen for this was a resource gathering.

The premise of this particular scenario was as follows: a certain amount of numbered resource markers (portrayed as flags in the simulation) were scattered across the simulated environment resembling a warehouse. The agents, starting from designated spawn points, are tasked with claiming as many of them as possible against a fixed time limit. \textit{Claiming} in this case was a process consisting of approaching a certain vicinity of a marker, at which point said marker was despawned and marked off as \textit{captured} by an agent initiating the claiming. Making the process instantaneous ensured that markers would be claimed on FCFS (\textit{First-Come, First-Served}) basis.

While the particulars were a subject to fine-tuning numerous times, Figure \ref{fig:x taskoverview} presents the intiial schema of agents' and markers' spawn points.
\begin{figure}[h]
    \centering
    \includegraphics[scale=0.2]{taskoverview}
    \caption{Initial positioning of agents' starting points and resource markers.}
    \label{fig:x taskoverview}
\end{figure}
\section{Incorporating Genetic Programming}
The Genetic Programming module was built on top of FRAIL's existing features, substantially changing the ususal flow of execution to make it suitable for learning processes. Initialisation, evaluation and reproduction steps are all done internally, with evaluation being a step when the actual simulations are run. Since the population was being tested sequentially and the maximum time one could run was 20s (with the possibility to end early, in case all markers were gathered before that time), some testing would take more then 48 hours to complete. A decision was made to increase update frequency of the simulations tenfold, thus guaranteeing simulations would complete afer maximum of 2s.
\subsection{Representation}
In the interest of avoiding integrity requirements and semantic incoherences (which would be undoubtedly caused by genetic operators), a specimen are instances of a specialized class, separated from the actual Behavior Tree implementation. These trees are then parsed during evaluation step to an actual Behavior Tree that is injected to an AI Controller in the simulation.

This simplified implementation allowed for easier manipulation of the trees' components. The nodes are distinguished by their type, but otherwise have all parameters required to operate. While this poses a serious data redundancy, the possibility of freely adding additional node types and ease of operations made this option highly desirable.
\subsection{Genetic Operations}
\subsubsection{Initialisation}
The population starts as a set of randomly generated specimen of varying size and depth. Each tree generation starts with a random Composite node in root, then the decision is made a at each further step to either add a random Action / Condition node to current root's children or recursively create a random subtree, all bound by a maximum size and maximum depth allowed. This approach provides a varied population while maintaining good balance of Action / Condition and Composite nodes.
\subsubsection{Selection}
The selection method of choice is a regular implementation of Tournament Selection. At each step, a random sample of a given size is selected from the current population and sorted according to a fitness value of the specimen. The first individual from this set is selected for Reproduction purposes.
\subsubsection{Crossover}
On each iteration, two selected specimen have a chance (determined by crossover rate parameter) to produce two children which will be put into the next generation instead of them. The crossover itself is a single-point version, implemented as a exchange of randomly selected nodes (one from each tree) without minding their type nor location.
\subsubsection{Mutation}
Both available mutation methods were implemented: initially, only the possibility for the tree to undergo micromutation is tested - that is implemented by resetting randomly chosen node's parameters to a random values. However, each tree selected for a micromutation process (probability of which is dictated by a mutation rate) has a chance to undergo a macromutation process instead - that, in turn, is determined by a separate, macromutation rate. In this case, micromutation step is replaced entirely with ``Headless Chicken'' mutation.
\subsubsection{Fitness function}
Due to format of the thesis, used fitness function varied in different testing scenarios. Detailed explanations of each will be included in the appropriate sections of Results chapter.


\chapter{Results}
\section{Cooperation Spectrum}
To learn how cooperative behaviors may influence success ratio in a particular task, the decision was made to first observe the evolution in case where agents work against one another, competing for a higher score. Only then, comparing the results with those of a cooperative case, conclusions regarding cooperation influence may be drawn.
The scenarios were tested in parallel with each change, were it to complicate the problem or put emphasis on certain components of the solutions.

The two scenarios shall be described in their dedicated sections below, focusing on key contrast points. Following, however are the elements common to both cases.
Evaluation of candidate solutions were done in the same environment, with identical starting points and resource markers' locations. In both scenarios, trees produced by Genetic Programming are accompanied by a unchanging, user-defined tree in the simulations. This reference agent uses one type of Action, \textit{goToFlag(flagNumber)}, visiting each resource marker in turn, capturing them from first to last, never changing path between simulations. Figure \ref{fig:x referenceagentdiagram} presents the Behavior Tree schematic of the reference agent. Existence of such tree could fill the function of both a pressuring component (reflected in the fitness function in the appropriate cases) and essentialy a time-gate, as gathering all markers would end the simulation. Furthermore, algorithms in both cases had access to identical Behavior Tree Components to build the trees from, thus ensuring equal sophistication of solutions in each scenario.

\begin{figure}[h]
    \centering
    \includegraphics[scale=0.4]{referenceagentdiagram}
    \caption{A scematic of reference agent used in simulations.}
    \label{fig:x referenceagentdiagram}
\end{figure}
Finally, the three components that partook in the specimen grading were as follows:
\begin{itemize}
    \item \textbf{Number of markers gathered} - a number of resource markers gathered.
    \item \textbf{Time to completion} - how much time did it take to finish the simulation (i.e. how much time has elapsed untill all the markers were claimed).
    \item \textbf{Size of a Tree} - how many nodes were in the tree.
\end{itemize}
% mention the task? the role of each component in producing optimal tree? define desirable solution?
% also, is it worth mentioning that the agents wouldn't know if the flag is claimed before getting there?
\section{Competitive Scenario Specification}
\subsection{Synopsis}
Competitive case assumed the optimal solution to the problem is maximizing the agent's personal gain. The markers gathered were counted separably for each agent, thus turning the task into a race - evolutionary agent was forced to get to as many markers as possible before the reference agent would claim them for himself.

After initial testing, the scenario was modified to feature 7 resource markers (instead of initial 5) and modified spawn points of agents -  readjusted to move evolutionary agent further back from the first marker. While increasing the number of markers served as an attempt to complicate the problem, moving evolutionary agent further from the first flag was done in the interest of being able to reproduce the results: since both agents were in the equal distance of the first flag and moved with the same speed, the matter of which one would claim the first flag was comparable to a coin-toss. Figure \ref{fig:x scenario1topdown} presents the final view of the competitive scenario map.

\begin{figure}[h]
    \centering
    \includegraphics[scale=0.2]{scenario1topdown}
    \caption{Finished map of the competitive scenario environment}
    \label{fig:x scenario1topdown}
\end{figure}

\subsection{Fitness Function} % should it include a plot of a fitness function?
The competitive scenario is effectively a zero-sum game. Considering that, the initial fitness value of the $i-th$ specimen in the population was a sum of $n$ markers gathered by an agent controlled by a tested tree, multiplied by a resource point constant $C_1$. This, however, wasn't feasible as a long-term solution. To introduce differentiability between the specimen scoring the same amounts of markers, a  factor of time in miliseconds multiplied by constant $C_2$ was then subtracted from the sum. Additionally, to coerce the algorithm into preferring smaller trees, a tree size factor with weight $C_3$ was further subtracted from this value. The finished formula for fitness function is presented on Equation \ref{eq:x scenario1fitness}.
\begin{equation}
    \label{eq:x scenario1fitness}
f(i) = C_1 * n - (\frac{t}{C_2} + \frac{s}{C_3})
\end{equation}
\subsection{Success Criterion}
With the goal of maximizing personal gain, which is the case with the considered scenario, desirable trees are those that claim every possible marker. However, since evolutionary agent's spawn position put it at the disadvantage (making it impossible to gather the first flag), only solutions with $n-1$ flags gathered out of $n$ possbile would be considered ``successful''. The true goal of a ideal solution was then to create a big enough difference in time taken to visit the markers, allowing it to reach and claim the last one.
\section{Cooperative Scenario Specification}
\subsection{Synopsis}
Cooperative scenario, in contrast to the ``selfishly'' motivated competitive one, takes a broader perspective. Instead of competing, agents were expected to work with each other towards claiming all 5 markers. However in time, the layout of the scenario was modified to maintain uniformity between the two cases' layout - that included both increasing the number of flags to 7 and modifying the agents' spawn positions - for comparison reasons.
\subsection{Fitness Function}
Due to competitive aspect being gone, a perspective on performance shifted substiantially. The number of markers claimed by an agent became an impractical characteristic, with the time factor providing a much more considerable view on specimen performance. However, the decision was made to include the \textit{total number of markers} in the fitness function, serving as a constant, to ensure that the results from both scenarios are of the same order of magnitude. Aforementioned time and tree size components remained in untouched form, all the more critical in case at hand. Equation \ref{eq:x scenario2fitness} presents the fitness function used in evaluating specimen in cooperative scenariio.
\begin{equation}
    \label{eq:x scenario2fitness}
f(i) = C_1 * (n + m) - (\frac{t}{C_2} + \frac{s}{C_3})
\end{equation} % legend?

\subsection{Success Criterion}
In the cooperative case, the underlaying goal is to divide the markers between two agents in the most efficient way. To effectively operate, the desired outcome is achieving completion time shorter than the time it takes for the reference agent to complete his route. In other words, a successful solution wille be able to successfuly adapt itself to a potential parter's route and act accordingly. % reword to make it clearer?

\section{Genetic Programming Parameters in Experiments}
After observing the results from a number of initial test runs as well as using \textit{grid-search} algorithm to help determine \textit{crossover rate}, \textit{mutation rate} and \textit{population size} parameters, the parameters used in the experiments are presented in table \ref{table:x selectedparameters}

\begin{table} [h]
    \centering
    \begin{tabular} {c c}
        \hline \hline
        Parameter Name & Parameter Value \\
        \hline
        Maximum Number of Generations & 100 \\
        Population Size & 200 \\
        Crossover Rate & 25\% \\
        Micromutation Rate & 15\% \\
        Macromutation Rate & 2\% \\
        Tournament Selection Tourney Size & 10 \\
        Starting Generation Minimum Tree Size & 20 \\
        Starting Generation Maximum Tree Size & 30 \\
        Maximum Tree Depth & 6 \\
    \end{tabular}
    \caption{Selected parameter values.}
    \label{table:x selectedparameters}
\end{table}

After a number inital runs, it was clear that keeping a maximum number of generations above one hundred served no function - there were no instances that did not converge before that point and following were long periods of stagnation or, even worse, loosing best specimen. The method responded positively to increasingly higher number of specimen in the generation: totalling at 200, this ensures a good variety of genetic material for the algorithm to make use of. Crossover and mutation rates were kept low due to the risk of destroying relatively low number of good specimen each generation. On the other hand, they could not have been set \textit{too low}, lest the algorithm would not explore the search space properly. It is worth pointing out that macromutation (``Headless Chicken'') was kept at extremely low occurence chance specifically for that reason - there weren't any instances suffering from too early convergence, something that macromutation would have been a remedy to. Choosing a 5\% value of tourney size ensured it would still be possible to get specimen with mid-range fitness value into the next generation while keeping the selection pressure relatively hight.

Tree related parameters were chosen after a careful consideration of desirable feats in the initial generation. The Behavior Trees in experiments were built from three kinds of nodes: \textit{Selector}, \textit{Sequence} and an Action \textit{goToFlag(flagNumber)}. Generating the tree between 20 and 30 nodes would create a varied populations with different shaped trees, while maintaining decent complexity bound by maximum tree depth.
\section{Exemplary Specimen}
In this section, a sample specimen representation is presented and described. The listing below presents the text representation of a tree, while figure \ref{fig:x exemplaryspecimenscenario1} contains a (arguably) more human-readable form.
\begin{lstlisting}
Sequence
  ActionGoToFlag 0
  ActionGoToFlag 3
  ActionGoToFlag 4
  ActionGoToFlag 4
  Selector
    Sequence
      Selector
        Sequence
          ActionGoToFlag 7
          ActionGoToFlag 5
          ActionGoToFlag 6
        ActionGoToFlag 6
      ActionGoToFlag 0
      ActionGoToFlag 3
      ActionGoToFlag 4
    ActionGoToFlag 3
    ActionGoToFlag 7
  Sequence
    ActionGoToFlag 2
    ActionGoToFlag 3
  ActionGoToFlag 4
  Selector
    ActionGoToFlag 1
    ActionGoToFlag 1
  ActionGoToFlag 5
  ActionGoToFlag 2
  ActionGoToFlag 3
\end{lstlisting}

\begin{figure}[h]
    \centering
    \includegraphics[scale=0.4]{exemplaryspecimenscenario1}
    \caption{Example diagram of a evolutionary generated specimen.}
    \label{fig:x exemplaryspecimenscenario1}
\end{figure}

This particular specimen was generated during one of competitive scenario experimentsand his evaluation placed it on value 5911. Knowing both the value of fitness value constants it can be proven that the specimen claimed 6 resource markers during a simulation lasting 8.7 seconds.

The tree has a very cluttered look and many unnecessary placed nodes, but the route should be clear: going from the top, it's going to visit the closest marker, which is the second one (sic!), then the third, proceeding to the fourth one (twice, in fact, but the second call will return instantly) and head for the last one, only to return to the fifth and sixth, probably meeting a reference agent nearby fifth marker. The rest of the nodes, as mentioned before, serve no function in this particular scenario. Unfortunately, the tree didn't follow the expected strategy (gaining a bit on each marker only to make a final push from sixth to seventh) instead opting to devise a new one. Although - as seen abole - the strategy of choice seems to be at least equally viable.
\section{Optimisation Results}
In this section, results of model cases (one from each scenario) will be presented. When analysing the results, three main components were kept in mind:
\begin{itemize}
    \item The shape of the algorithms' best specimen fitness value and its relation to average fitness value in current iteration.
    \item An average size of a tree in the current population.
    \item A number of iterations needed for the best specimen to achieve success.
\end{itemize}
The shape of fitness plots indicate the general state and health of the Genetic Programming algorithm. While the plot indicating top fitness value in the population is certainly important (the success measure is directly tied to a top specimen, after all), it's the average fitness value in the population that provides the much needed information on population growth, stagnation or even oscillations, indicating local plateaus.

The average size of a tree is also a useful indicator of a general perceived value of the generation: while larger trees are not neccessairly always worse (since the execution time is virtually neglible), they most definietly contain needless constructs. Apart from being nigh unreadable, disregarding tree size would certainly result in need of introducing some kind of post-processing to prune a number of never-to-be-executed branches.

Finally, the number of iterations needed to produce a first solution that is deemed ``successful''. Looking into the matter from the user's perspective,the only feasible characteristic is the speed of achieving a \textit{fit} specimen, able to deal with the task at hand.

Figure \ref{fig:x competitivefitnessplot} presents a plot of the highest fitness value and average fitness in the population in respect to iteration in the competitive scenario. Figure \ref{fig:x competitivetreesizeplot}, hovever, illustrates the change of average tree size in the population in the course of the algorithm's run in the same scenario.

\begin{figure}[h]
    \centering
    \includegraphics[scale=0.6]{competitivefitnessplot}
    \caption{A fitness value with respect to iteration plot in the competitive scenario.}
    \label{fig:x competitivefitnessplot}
\end{figure}

\begin{figure}[h]
    \centering
    \includegraphics[scale=0.6]{competitivetreesizeplot}
    \caption{An average tree size with respect to iteration plot in the competitive scenario.}
    \label{fig:x competitivetreesizeplot}
\end{figure}
The algorithm's run in this particular case started on a relatively high note, with the best specimen claiming 4 markers for himself. In the next few iterations, best fitness achieved increased, then oscillated around 4200 value, to finally stabilize on a value closer to 5000. Up to this point, one can see the average fitness in the generation rising, indicating that, while there were no radical increases in value, the algorithm is producing more and more feasible options. Then, surely due to particularly good mutation or crossover, a breakthrough is made and the population achieves its first successful specimen. This one, however, is instantly lost, having not been copied over to the next generation. The curve then starts oscillating heavily, with the algorithm producing further successful specimen only to lose them few cycles later. Finally, around iteration 50 mark, the best specimen is kept for a little longer. Note that the first successful individual marked the decline in average fitness growth. That proves that by this time, the population has been saturated with useful gene structures.

The size plot is intended to be considered and analysed together with the previous one. Having done that, one can observe that with the first successful specimen produced, the population started intensively increasing in node count. Having achieved large enough pool of fit solutions, size started to be the diffrentiating factor, successfully lowering the number. This is the perfect example of larger trees being promoted in the initial stages (since they have a larger chance of containing a useful node structure), but having their value decreased as the useful genes from them are passed into further generations. This also poses an interesting question - whether the first found successful specimen should be accepted, or will processing more generations rewarding?

\begin{figure}[h]
    \centering
    \includegraphics[scale=0.6]{cooperativefitnessplot}
    \caption{A fitness value with respect to iteration plot in the cooperative scenario.}
    \label{fig:x cooperativefitnessplot}
\end{figure}

\begin{figure}[h]
    \centering
    \includegraphics[scale=0.6]{cooperativetreesizeplot}
    \caption{An average tree size with respect to iteration plot in the cooperative scenario.}
    \label{fig:x cooperativetreesizeplot}
\end{figure}

Having seen the competitive scenario, the fitness plot in figure \ref{fig:x cooperativefitnessplot} might not appear as telling at first. Having produced a successful specimen in the first generation, the individual is instantly destroyed, its genes dilluted in the next generation. Another one, however, is produced as quickly as the fourth iteration of the algorithm. The value then remains virtually unchanged, oscillating only slightly to the last generation. Average fitness value curve reflects that progress well, with the oscillations starting only when the other one stabilizes. 

The feeling disappears, however, looking at the average tree size change presented in figure \ref{fig:x cooperativetreesizeplot}. On 30th generation, when the fitness value of the best specimen appears to enter stagnation, abrupt decrease of an average specimen node count begins. Over the next 50 iterations of the algoritm, the average size is reduced almost thrice - from 36 nodes to 13. 


\chapter{Conclusions and Future Work}
The aim of this thesis was answering the research question:  ``How cooperative behaviors between artificial agents influence the performance in multi-agent task-oriented environments?''. On the grounds of success criteria defined in the previous sections: the increase of speed in reaching the solution, the size of a successful candidate and speed of task execution itself all testify to cooperative behaviors having a considerable influence in tested task. Specimen in the cooperative scenario achieved quicker completion time compared to those evolved in the contrasting, competitive one. Furthermore, later iterations in the cooperative case would feature specimen of much smaller size executing the task with equal effectiveness.

These results prove that cooperative behaviors are, in fact, \textit{feasible} to evolve.
With that said, the work here merely touches the surface of genetically evolving Behavior Trees and the behavior types they produce. Future work will focus at deeper analysis of the population, with aim to possibly make genetic operators work with Behavior Tree representation seamlessly. Developing more sophisticated methods of crossover and mutation, aware of constraints of Behavior Tree encoding, would doubtlessly increase the effectiveness of evolving trees. Furthermore, the concept of adaptation of the agent to a given environment should be explored. The task considered in this thesis - arguably simple - has ben proven to be solvable with the simpliest of elements. A natural next step would be increasing both the dificultness of the problem as well as the resources available to the algorithm.


\listoffigures
\printbibliography
\end{document}
